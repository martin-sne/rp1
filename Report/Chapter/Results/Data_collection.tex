\subsection{Data collection}
\label{section:data-collection}
Collecting data from zones is not an easy task, especially when the amount of available time is relatively small. This is the reason why this assignment has been divided in two Python scripts\footnote{The scripts are data\_wrapper\_JN.py and data\_wrapper\_ML.py}, each developed by one of us. The first script has two purposes:
\begin{itemize}
\item Creating an XML template \textit{data.xml}
\item Collecting and inserting DNSSEC data into a new XML file (\textit{updated.xml}) which is based on \textit{data.xml}
\end{itemize} 

\noindent The second script focuses on updating \textit{updated.xml} with other DNSSEC data. It is worth noticing that our proof of concept is still at a beta level and needs to be enhanced. For instance, the scripts written for data retrieval fail if the name server of a zone is down. That said, let's assume they are all up.

\subsubsection{General data structure}
For security and scalability reasons, the SNMP subagent cannot fetch the data directly from the monitored zones. Indeed, some of the checks the scripts perform are expensive\footnote{Verifying the DNSKEY RRset against the published KSK involves the \textit{Crypto} Python module.} and data retrieval would have became a real burden for the SNMP subagent when dealing with a large number of domains. It must retrieve the data from a central repository instead. \textit{XML} (EXtensible Markup Language) seems to be the appropriate choice as it is meant to represent data-structures in a human and machine readable format, where the information is easily accessible for applications. However, the use of XML might affect performances when dealing with a large number of zones. This has not been witnessed during this project though, as we dealt with four zones only.
\\
We used Python and the \textit{ElementTree} XML API\cite{et} to build an XML template describing for each zone the MIB objects and their syntax. The resulting XML document is shown in listing \ref{listing:xml}


\begin{listing}
\begin{verbatim}

<?xml version="1.0" encoding="UTF-8" ?>
<ZoneList>
  <Zone id="1" name="berlin.warsaw.practicum.os3.nl">
    <table name="dnssecZoneGlobalTable">
      <item>
        <data id="2" name="dnssecZoneGlobalServFail" type="Integer32"> </data>
      </item>
      [...]
    </table>
    [...]
  </Zone>
  [...]
</ZoneList>
\end{verbatim}
\caption{XML template}
\label{listing:xml}
\end{listing}
The main element of the file is \textit{ZoneList}, which is divided into \textit{Zone} sub-elements. Each zone element has two attributes which are \textit{id} and \textit{name} defining respectively the zone id and its name. The former is incremented for each new zone element, and the latter is retrieved from a text file\footnote{The text file "zone\_hint" is made of key/value pairs (zone/IP). The hard-coded IP address is used only if the corresponding zone shows a SERVFAIL status.} that contains zones to monitor. The \textit{Zone} element is split into sub-elements as well, \textit{table}. \textit{table} has one attribute \textit{name} which values are tables defined in the MIB module, namely \textit{dnssecZoneGlobalTable}, \textit{dnssecZoneSigTable}, \textit{dnssecZoneAuthNSTable}, \textit{dnssecZoneDiffTable}. Each table is divided into sub-elements \textit{item}. An \textit{item} contains a sub-element \textit{data} with three attributes, \textit{id}, \textit{name} and \textit{type}. These represents the objects of the tables defined in the MIB module. The \textit{id} is the OID number of the corresponding table described in the \textit{name} attribute, and \textit{type} is the syntax of the content between the \textit{data} tags. The content is empty so far as it is going to be filled in with DNSSEC data of monitored zones, retrieved by a later part of the script.

\subsubsection{Retrieving DNSSEC data}
The XML template is now created and needs to be filled in with DNSSEC data. First of all, we verify whether the zone we want to retrieve data from is available from a resolver point of view in order to make sure that the DNSSEC data can be accessed. If it is not\footnote{The zone shows a SERVFAIL status probably because the signature of at least one of its apex RR is expired.}, we query its authoritative name server instead. The query consists of either an AXFR or a DNS query made possible thanks to the \textit{dnspython} library that allows to retrieve the data described in section \ref{section:vital-life-signs-dnssec}. Furthermore, a copy of \textit{data.xml}, named \textit{updated.xml}, is created to avoid read/write conflicts. The data freshly retrieved is inserted into it by means of a Python dictionary and XPath queries. The values of each item for each table represents a key in the dictionary and the retrieved data is the corresponding value as shown in Listing \ref{listing:dic}. Since the keys are constructed in a specific way ("zone\_name" + "\_" + "table" + "\_" + "item\_ID"), the XPath queries allow to navigate through the XML document in order to tell the script where to add DNSSEC data at the right place in the \textit{updated.xml} file. 
\begin{listing}
\begin{verbatim}
{'warsaw.practicum.os3.nl_dnssecZoneSigTable_4': '"20161212121212"',
'derby.practicum.os3.nl_dnssecZoneDiffTable_2': '1',
'warsaw.practicum.os3.nl_dnssecZoneSigTable_3': '"20161212121212"',
'warsaw.practicum.os3.nl_dnssecZoneSigTable_2': '"20150112142419"', 
[...]}
\end{verbatim}
\caption{Python dictionary}
\label{listing:dic}
\end{listing}
Moreover, we have specified integer values that the scripts return for some checks. They are defined as textual conventions in the MIB and are associated to a specific state as described in table \ref{table-tc}. For instance, one could say that according to Listing
\ref{listing:dic}, the zone file for the \textit{derby.practicum.os3.nl} zone is the
same on the master and on slave name servers as serial numbers do not
differ\footnote{\textit{dnssecZoneDiffTable\_2} represents the second item of the
\textit{dnssecZoneDiffTable} which checks for discrepancies in zone file serial
numbers between the master and the slave name servers.}.

%The \textit{AXFR} allows to get the entire zone data that can be easily parsed with the python \textit{re} module. Then, we can retrieve for a zone the number of resource records, the minimum and maximum TTL, the number of DS records and delegations and compare them with each other, the expiration date for the SOA, NS and DNSKEY RRSIG, the inception date of the oldest signature, and the discrepancies between serial numbers provided by the master and slave name server. The DNS queries help us to fetch the SOA RR, the name servers and their IP and  to validate the DNSKEY RRset with the published KSK,    

