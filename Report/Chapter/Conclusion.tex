\section{Conclusion}
\label{chap:conclusion}
Our research project has been divided into several steps in order to build a proof of concept that can be implemented to monitor DNSSEC through SNMP. First of all, we have defined what are the vital life signs of a DNSSEC zone, i.e. the variables that are essential to the proper functioning of such a zone. Then, we have constructed a Management Information Base module for SNMP, located under the ARPA2 OID tree, that delivers the variables previously defined for each zone that is monitored. In addition, we have written an SNMP subagent in Python based on the \textit{python-netsnmpagent} module, that communicates with the NET-SNMP master agent and updates the SNMP objects. We also have written Python scripts that collect DNSSEC data for each monitored zone into a central XML file easily accessible by the SNMP subagent. A new zone can be monitored automatically just by adding its domain name with its corresponding name server IP address in the \textit{zone\_hint} file. Finally, we have conducted monitoring based on our proof of concept using the Nagios monitoring system. Other monitoring architectures for DNSSEC employ different ways to retrieve DNSSEC data such as remote procedure calls. As SNMP is a standard application protocol, our proof of concept can be deployed easily to monitor DNSSEC enabled zones. However, it has been tested with only four zones, and might face performance issues when dealing with a large number of zones for several reasons, inter alia, the adoption of XML, AXFR requests to retrieve data and expensive checks (e.g. usage of the Python \textit{Crypto} module).
\\
The DNSSEC-MIB file, the subagent and all related software components are available on Github \cite{github}.


